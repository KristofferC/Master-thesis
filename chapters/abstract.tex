% Sample file on how to use subfiles.
\documentclass[abstract.tex]{subfiles}

\begin{document}

The aim of this thesis it to give a method of how to include grain boundary mechanisms into models of microstructures containing grains of polyhedral shape.
First, microstructures of this type is generated by Voronoi tesselation. FE analyses simulating uniaxial tensile testing of cubic RVEs with these microstructures are then performed for grains with different size. The results of these analyses showed a decrease in the variance of the stress response among analyses with grains of the same size. An increase of the stress at the same strain could also be seen for microstructures with higher number of grains.  The second part describes the method used in this thesis to include grain boundary mechanisms in the model. This is achieved by inserting cohesive elements that models a traction separation law between the grains. FE analyses comparing implicit and explicit solvers were performed on microstructure and it was found that using the implicit solver gave convergence problems an explicit solver could be used to overcome this. Last, a traction separation law generated from atomistic simulations that represent the grain boundary mechanism between tungsten carbide infiltrated by cobalt was used to estimate the yield stress of this material. The results showed a yield stress around $\unit[MPa]{350}$ but more extensive analyses are required to give confidence to this result.
\end{document}
