% Sample file on how to use subfiles.
\documentclass[abstract.tex]{subfiles}

\begin{document}

The aim of this thesis it to give a method  to include grain boundary mechanisms into models of microstructures containing grains of polyhedral shape.
First, microstructures of this type is generated by Voronoi tesselation. Then FE analyses simulating uniaxial tensile testing of cubic RVEs with these microstructures were performed for a crystal plasticity model with grains of different sizes. The results of these analyses showed a decrease in the variance of the stress response when the number of grains was increased. An increase of the stress at the same strain could also be seen for microstructures with higher number of grains.  The next part describes the method used to include grain boundary mechanisms in the model. This is done by inserting cohesive elements between the grains. The cohesive behavior of the cohesive elements is defined by a traction separation law. FE analyses on models containing cohesive elements were performed using both an implicit and explicit time integration technique to solve the FE equations. It was found that when using the implicit solver it was difficult to complete the analyses due to convergence problems. The explicit solver did not have these problems and gave the same results as the implicit solver at the convergence region for the implicit solver. Last, a traction separation law was adapted from from atomistic simulations that represent the grain boundary mechanism between tungsten carbide infiltrated by cobalt. The three dimensional grain structure model was used to estimate the yield stress of this material. The results indicated a yield stress around $\unit[350]{MPa}$ but more extensive analyses are required to give confidence to this result.
\end{document}
