% Sample file on how to use subfiles.
\documentclass[abstract.tex]{subfiles}

\begin{document}

The aim of this thesis it to give a method  to include grain boundary mechanisms into models of microstructures containing grains of polyhedral shape.
First, microstructures of this type is generated by Voronoi tesselation. FE analyses simulating uniaxial tensile testing of cubic RVEs with these microstructures was then performed for a crystal plasticity model with grains of different sizes. The results of these analyses showed a decrease in the variance of the stress response where the number of grains was increase. An increase of the stress at the same strain could also be seen for microstructures with higher number of grains.  The second part describes the method used in this thesis to include grain boundary mechanisms in the model. This is achieved by inserting cohesive elements that models a traction separation law between the grains. FE analyses on models containing cohesive elements were performed using both an implicit and explicit solver to solve the FE equations. It was found that when using the implicit solver it was difficult to get complete analyses due to convergence problems but the explicit solver did not have these problems and gave the same results as the implicit solver at the convergence region for the implicit solver. Last, a traction separation law generated from atomistic simulations that represent the grain boundary mechanism between tungsten carbide infiltrated by cobalt was used to estimate the yield stress of this material. The results showed a yield stress around $\unit[350]{MPa}$ but more extensive analyses are required to give confidence to this result.
\end{document}
