% Sample file on how to use subfiles.
\documentclass[meshing_micro.tex]{subfiles}

\begin{document}

\chapter{Meshing of microstructure}

In order to do a finite element analysis the generated microstructure need to be meshed. One way of creating such a mesh, which would be computationally fast and simple to implement, would be to ignore the shape of the grains and use a structured mesh of hexahedral elements. Each element would then be said to be a part of that grain which occupies most of its volume. With a reduction in the size of the hexahedrons the shapes of the grains are better approximated. Example of such meshes can be seen in figure \ref{fig:pois_voronoi_mesh_cube}.

\begin{figure}
\centering
\begin{subfigure}[b]{.5\textwidth}
  \centering
  \includegraphics[width=.5\linewidth]{./figures/img_body_mesh_cube}
  \caption{}
  \label{fig:pois_voronoi_mesh_cube_a}
\end{subfigure}%
\begin{subfigure}[b]{.5\textwidth}
  \centering
  \includegraphics[width=.5\linewidth]{./figures/img_body_mesh_cube_fine}
  \caption{}
  \label{fig:pois_voronoi_mesh_cube_b}
\end{subfigure}
\caption{Structured mesh of different coarseness.}
\label{fig:pois_voronoi_mesh_cube}
\end{figure}

A refinement of having each hexahedral finite element belong to one single grain would be to not let all the Gauss points in the element be considered a part of the same grain. 
If an element is intersected by two or more grains the Gauss points inside that element could be assigned to the grain they are spatially located in. 
This has been done in \cite{Nygards20031049}, \cite{Cailletaud2003351} and \cite{Barbe2001513}. There are however drawbacks to using a structured mesh. 
In an unstructured the exact shape of the grains can be respected with a few number of elements. In some material models a fine mesh of the grains are not needed but the shape of the grains might still be important to approximate well. If a structured mesh is used this puts an upper bound on the size of the elements and thus require a larger computational time for the same analysis to be performed.
In some cases, for example \cite{Bohlke201011} the result is not significantly different when using a structured or unstructured mesh, even if the same number of degrees of freedom is used in the analysis. However, in other cases such as \cite{Li20091163} it was found that when calculating the elasto-viscoplastic response of polycrystalline microstructures they could reduce the number of elements and still get a convergent result if they used a conforming unstructured mesh. 

To generate the mesh the software Neper was used. 
This is the same software that was used to generated the Voronoi tessellation. Neper has support of generating a conforming mesh which was used in this thesis. Neper uses a scheme to generate a mesh in which it starts meshing in the lowest dimension and then uses these elements as seed to generate element of a higher dimension  In practice this  node is created at all vertices. Connecting these nodes to each other the edges are created and can be meshed by subdividing them into smaller intervals. 
By connecting the edges the faces are created and they are meshed using triangles. Finally, the faces are connected, creating the grains, which are meshed with tetrahedrons using the triangles on the faces as seeds. A figure illustrating the process can be seen in \ref{fig:mesh_strat}. There are many different algorithms used to do the actual meshing from the lower dimensional seeds. 
What Neper does is that it uses different algorithms in parallel and from the results chooses the one that generates the mesh with overall highest quality.

 \begin{figure}
\centering
  \includegraphics[width=.85\linewidth]{./figures/meshing_tact.pdf}
\caption{Bottom up meshing strategy used by Neper. Grains are meshed from lower to higher dimensions.}
\label{fig:mesh_strat}
\end{figure}


One problem in meshing a Voronoi tessellation is that the tessellation is likely to contain faces that are smaller than the desired characteristic size of the finite volume elements that make up the mesh. One solution to this is to have the mesh more refined in those parts of the model but this will lead to a significant increase in the total number of elements in the mesh.  Another method is to slightly alter the shape of the grains in such a way that the small faces and edges are removed but the general shape of the grain is preserved. It is likely this will not significantly alter the results of the analysis. The latter method is what the Neper software use. This can be seen if one carefully compares the unmeshed microstructure in figure \ref{fig:pois_voronoi} to the meshed in figure \ref{fig:pois_voronoi_mesh}. Some of the small faces and edges in the unmeshed case have been removed in the meshing process. A detailed description of the algorithm used to reshape the grains can be read in the referenced paper.
 
The output file from Neper is a file that lists all the nodes in the mesh and the elements in terms of the connectivity of the nodes. What is also given in the file that different sets of nodes and elements. For example, for every grain there is a set of what elements that are part of that grain. This means that in the analysis later it is easy to assign a certain material to a certain grain by simply letting all elements in that grains element set be of that material. The node sets are sets of the nodes that are in the different boundaries of the cube. The nodes in the bottom face are for example in their own set. This means that boundary conditions on the faces of the cubes can easily be set.
\end{document}
