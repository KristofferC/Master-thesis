% Sample file on how to use subfiles.
\documentclass[meshing_micro.tex]{subfiles}

\begin{document}

\chapter{Meshing of microstructure}

In order to perform a finite element analysis the generated microstructure needs to be meshed. One way of creating such a mesh, which would be computationally fast and simple to implement, would be to ignore the shape of the grains and use a structured mesh of hexahedral elements. Each element would then be said to be a part of that grain which occupies most of its volume. With a reduction in the size of the hexahedrons the shapes of the grains are better approximated. Example of such meshes can be seen in figure \ref{fig:pois_voronoi_mesh_cube}.

\begin{figure}
\centering
\begin{subfigure}[b]{.5\textwidth}
  \centering
  \includegraphics[width=.5\linewidth]{./figures/img_body_mesh_cube}
  \caption{}
  \label{fig:pois_voronoi_mesh_cube_a}
\end{subfigure}%
\begin{subfigure}[b]{.5\textwidth}
  \centering
  \includegraphics[width=.5\linewidth]{./figures/img_body_mesh_cube_fine}
  \caption{}
  \label{fig:pois_voronoi_mesh_cube_b}
\end{subfigure}
\caption{Structured mesh of different coarseness.}
\label{fig:pois_voronoi_mesh_cube}
\end{figure}

A refinement of having each hexahedral finite element belonging to one single grain would be allow the Gauss points in one element to belong to different grains. 
If an element is intersected by two or more grains the Gauss points inside that element could be assigned to the grain they are spatially located in. 
This method has been adopted \cite{Nygards20031049}, \cite{Cailletaud2003351} and \cite{Barbe2001513}. There are however drawbacks to using a structured mesh. 
In an unstructured the exact shape of the grains can be respected with a few number of elements. In some RVE:s a fine mesh of the grains are not needed but the shape of the grains might still be important to approximate well. If a structured mesh is used this puts an upper bound on the size of the elements and thus requires a longer computational time.
In some cases, for example \cite{Bohlke201011} the results are not significantly different when using a structured or unstructured mesh, even if the same number of degrees of freedom is used in the analyses. However, in other cases such as \cite{Li20091163} it was found that when calculating the elasto-viscoplastic response of polycrystalline microstructures, the number of elements could be reduced and still get a convergent result when using a conforming unstructured mesh. 

To generate the mesh the software Neper\cite{Quey20111729} was used. 
This is the same software that was used to generated the Voronoi tessellation. Neper has support of generating a conforming mesh which has been used in this thesis. Neper uses a scheme to generate a mesh which starts with meshing in the lowest dimension and then uses these elements as seeds to generate element of a higher dimension.  In practice, first nodes are defined at all vertices. Then these nodes are connected along the edges where new nodes are defined at desired intervals. 
By connecting the edges with each other the faces are created and these are meshed using triangles. The nodes in the edges surrounding the face is used as seeds for these triangles. Finally, the faces are connected, defining the grains, which are meshed with tetrahedrons using the triangles on the faces as seeds. A figure illustrating the process is shown in figure \ref{fig:mesh_strat}. There are many different algorithms used to do the actual meshing from the lower dimensional seeds. 
What Neper does is that it uses different algorithms in parallel and from the results chooses the one that generates the mesh with the overall highest quality.

 \begin{figure}
\centering
  \includegraphics[width=.85\linewidth]{./figures/meshing_tact.pdf}
\caption{Bottom up meshing strategy used by Neper. Grains are meshed from lower to higher dimensions.}
\label{fig:mesh_strat}
\end{figure}


One problem in meshing a Voronoi tessellation is that the tessellation is likely to contain faces that are smaller than the desired characteristic size of the finite elements. One solution to this is to refine the mesh in those parts of the model but this will lead to a significant increase in the total number of finite elements.  Another method is to slightly alter the shape of the grains in such a way that the small faces and edges are removed but the general shape of the grain is preserved. It is likely this will not significantly alter the results of the analysis. The latter method is what the Neper software usse. This can be observed if one carefully compares the unmeshed microstructure in figure \ref{fig:pois_voronoi} to the meshed microstructure in figure \ref{fig:pois_voronoi_mesh_cube}. Some of the small faces and edges in the unmeshed case have been removed during the meshing process. A detailed description of the algorithm used to reshape the grains is given in \cite{Quey20111729}.
 
The output file from Neper lists all the nodes in the mesh and the elements in terms of the connectivity of the nodes. In addition, the output file defines different sets of nodes and elements. For example, for every grain there is a set with the elements that are part of that grain. This means that in the analysis later it is easy to assign certain material properties to each grain. The node sets are sets of the nodes that are found on different boundaries of the cube. The nodes on the bottom face of the RVE are for example listed in their own set. This means that boundary conditions on the faces of the RVE:s can easily be set.
\end{document}
