% Sample file on how to use subfiles.
\documentclass[crystal_plast.tex]{subfiles}

\begin{document}

\FloatBarrier

\chapter{Crystal plasticity in duplex steel}
\label{chap:cryst_plast}

A material model based on crystal plasticity adopted for a two phase duplex steel has already been developed in \cite{lillekh}. Since it is well described there only the parameter values to the material model will be given here. They can be seen in table \ref{tbl:mat_par}. The analyses previously performed with that material model have only been using a two dimensional RVE:s assuming plain stress conditions. It is therefore interesting to see how the same model behaves for a three dimensional RVE and study if significant changes in the stress strain curves are obtained for similar loading conditions.


\begin{table}[htpb!]
\caption {Material parameters used in the analysis.}
\label{tbl:mat_par}
\centering
    \begin{tabular}{cccccccccccc}
    \toprule%
    \multirow{2}*{Material}      & G & $\lambda$ & $\tau^y$  & \multirow{2}*{q}   & \multirow{2}*{$\xi$} & $h_0$  & $h_\infty$  & \multirow{2}*{$\eta$} & \multirow{2}*{$m$} & \multirow{2}*{$n$} & $t_*$ \\ 
        &  [GPa] & [GPa] & [MPa] &    &  &  [MPa] & [MPa] & &  &  & [s] \\\otoprule%
    Austenite & 71      & 106            & 210            & 0.1 & 75    & 4000        & 0                & 0.0001 & 1.0 & 1.0 & 50        \\ 
    Ferrite   & 71      & 106            & 250            & 1.1 & 50    & 100         & 0                & 0.001  & 1.0 & 1.0 & 50        \\\bottomrule
    \end{tabular}
\end{table}



\section{Loading case}

The FE analyses performed simulated an RVE (cube) inside a larger structure experiencing a uniaxial tensile testing. This is done by prescribing suitable boundary conditions on the RVE. The boundary conditions are defined by prescribing the displacements of certain nodes of the model. To be specific, the nodes at the bottom face of the cube are fixed such that they cannot move in the $y$-direction. The nodes at the bottom face on two of the edges have the additional boundary condition that they can only move parallel to their edge, i.e. $u_x = 0$ at $z = 0$ and $u_z = 0$ at $x = 0$. Stresses in the model are induced by having the nodes in the top face of the cube move at a fixed rate in the $y$-direction. An illustration of the boundary conditions is shown in figure \ref{fig:test_case_cryst}.

\begin{figure}[htpb!]
\centering
\includegraphics[width=0.6\linewidth]{figures/test_case}
\caption{Illustration of the boundary conditions applied to the RVE.}
\label{fig:test_case_cryst}
\end{figure}

In order to perform an FE analysis each element in the model needs to be assigned properties that governs the stress strain behavior of the specific element. In the FE analysis to be performed we want to be able to assign certain material properties depending on what grain the element belongs to. In the meshed model a grain is just a collection of finite elements. These collections are available as element sets so that instead of having to assign the material to each and every finite element it is sufficient to assign the properties to the element set that represent one grain.

In this thesis the two phases are assigned to random grains such that equal volume fractions are obtained. Each grain is then assigned random slip directions. A figure of the RVE after the material properties have been assigned to the grains is given in figure \ref{fig:ausferr}. 


 \begin{figure}[htpb!]
\centering
\begin{subfigure}{.5\textwidth}
  \centering
  \includegraphics[width=.75\linewidth]{./figures/250_grains_ausferr}
  \caption{}
  \label{fig:ausferr_3d}
\end{subfigure}%
\begin{subfigure}{.6\textwidth}
  \centering
  \includegraphics[width=.5\linewidth]{./figures/2d_voro_ausferr}
  \caption{}
  \label{fig:ausferr_2d}
\end{subfigure}
\caption{a) An example of a three dimensional microstructure with austenite and ferrite assigned to the grains.  b) Two dimensional grain structure used in \cite{lillbacka2007multiscale}, reprinted with permission.}
\label{fig:ausferr}
\end{figure}


%\subsection{Calculation of average stress}
%We want the volume averaged stress in the $z$ direction. This could be done by summing the stress from all the elements:
%
%\[  \overline{\sigma_{zz}} = \frac{1}{V} \int_\Omega \sigma_{zz} \id V \]
%
%This would however require saving of data for all the elements in the model. It is possible to extract the same information by only saving the force in the
%nodes of the top boundary of the cube. The following calculation justifies this. Using the divergence theorem the following relation can be written.
%$ \int_\Omega ( \sigma_{ij} x_k )_i \id x = \int_\Omega \sigma_{ij,i} x_k + \overbrace{\sigma_{ij} \underbrace{x_{k,i}}_{\delta_{ki}}}^{\sigma_{kj}} \id V $
%\begin{equation}
%\int_\Omega \sigma_{ij,i} x_k \id x = \int_\Gamma \sigma_{ij}  x_k \hat{n}_i \id s
%\end{equation}
%Using the product rule we can write
%\begin{equation}
%\int_\Omega \sigma_{ij,i} x_k \id x = \int_\Gamma \sigma_{ij}  x_k \hat{n}_i \id s
%\end{equation}
%\begin{equation}
%\int_\Omega ( \sigma_{ij} x_k )_i \id x = \int_\Omega \sigma_{ij,i} x_k + \overbrace{\sigma_{ij} \underbrace{x_{k,i}}_{\delta_{ki}}}^{\sigma_{kj}} \id V 
%\end{equation}
%\begin{equation}
%= \int_V \sigma_{kj} \id V = V \overline{\sigma_{kj}}
%\end{equation}
%\[ \overline{\sigma}_{zz} = \frac{1}{V} \int_\Omega \sigma_{zz} dV = \frac{1}{A} \int_\Gamma F_{z} dA? \]



\section{Results}

Five finite element analyses were conducted for models with different number of grains. The number of grains used in the different models were 5, 20, 50, 100, 250 and 500. The difference between analyses with the same number of grains is in the randomness of the seed points in the Voronoi tesselation and the random assignment of crystal slip directions in each grain.  The nodes on the upper boundary of the cube were displaced at a rate that corresponds to a change in strain of $\unit[3.125 \cdot 10^{-3}]{s^{-1}}$ until a total macroscopic strain of 0.05 was obtained. In order to highlight the heterogeneity at the mesoscopic level the accumulated plastic slip and the von Mises equivalent stress field of an RVE is shown for one analysis in figure \ref{fig:ausferr_col}. 

\begin{figure}[htpb!]
\centering
\begin{subfigure}{.8\textwidth}
  \centering
  \includegraphics[width=.8\linewidth]{./figures/250_ausferr_stress}
  \caption{}
  \label{fig:ausferr_stress}
\end{subfigure}%

\begin{subfigure}{.8\textwidth}
  \centering
  \includegraphics[width=.8\linewidth]{./figures/250_ausferr_plast}
  \caption{}
  \label{fig:ausferr_plast}
\end{subfigure}
\caption{(a) The von Mises equivalent stress field and (b) the accumulated plastic slip for a mesomodel with 100 grains.}
\label{fig:ausferr_col}
\end{figure}


The resulting stress strain curves can be seen for the different analyses in figure \ref{fig:sep}. From this figure there are a few observations that can be made. Before any plastic strain develops all the different analyses have exactly the same stress strain response. This is to be expected since at this amount of strain, the grains behave elastically and the crystal orientation has no effect. Therefore the whole model can be seen as one homogeneous cube. It can also be noted that the variance between the analyses with the same number of grains is reduced when the number of grains is increased. In a model with more grains, the heterogeneity in the model tends to cancel out giving a more representative result for a macroscopic model.


 \begin{figure}[ht]
\centering
\includegraphics[width = 0.65\textwidth]{figures/sep}
\caption{5 realizations (analyses) where slip directions are randomly varied of mesomodels with 5, 20, 50, 100, 250 and 500 grains.}
\label{fig:sep}
\end{figure}

The average stress strain responses for the analyses with the same number of grains are compared for different number of grains figure \ref{fig:avg}.
\begin{figure}[htpb!]
\centering
\includegraphics[width = 0.65\textwidth]{figures/avg}
\caption{Average responses of 5 different realizations for 5, 20, 50, 100, 250 and 500 grains. The experimental result found in \cite{lillekh} for the two-phase stainless steel is given.}
\label{fig:avg}
\end{figure}
It can be seen that the stress values tend to increase with more grains in the model and a satisfying convergence is not achieved. In order to determine the cause of this more analyses were conducted. To reduce the heterogeneity these analyses were performed with only ferrite in the RVE. Analyses with up to 2000 grains were conducted. The boundary conditions were the same as in the analyses that used a 50\% austenite and ferrite ratio. The same general trend could be seen in these FE analyses, the stress response is slightly stiffer with increasing number of grains. To quantify this effect the stress at a macroscopic strain of 0.05 was plotted for the different analyses with respect to the effective sphere diameter $\overline{D}_g$ defined as
\begin{equation}
 \overline{D_g} = 2\cdot\left( \frac{ 3\overline{V}}{4 \pi} \right)^{1/3},
\end{equation}
where $\overline{V}$ is the average volume of a grain. The result is shown in figure \ref{fig:avg_sphere_rad}. A line is also shown in the figure where the two parameters $\sigma_0$ and $k_y$ have been fitted to the data to best fit the curve
%
\begin{equation}
\label{eq:hallpetch}
 \sigma_{zz} = \sigma_0 + \frac{k_y}{\sqrt{D}}
\end{equation}
with the results $\sigma_0 \approx 650$ and $k_y \approx 27$. Equation \ref{eq:hallpetch} is the Hall-Petch relation. It is not yet clear if the material model follows this relation but what is clear is that there is a relation between the grain size and the stress in the model for equal macroscopic strains. Further investigations must be conducted to understand the reason why the averaged stress do not converge when increasing the number of grains.


\begin{figure}[htpb!]
\centering
\includegraphics[width = 0.65\textwidth]{figures/avg_fig_sphere_rad}
\caption{The stress response at a macroscopic strain of 0.05 against the equivalent sphere diameter.}
\label{fig:avg_sphere_rad}
\end{figure}

\end{document}
