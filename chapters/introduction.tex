% Sample file on how to use subfiles.
\documentclass[introduction.tex]{subfiles}

\begin{document}

\chapter{Introduction}



 \begin{figure}
\centering

\begin{subfigure}{.1\textwidth}
  \centering
  \includegraphics[width=.2\linewidth]{./figures/macro.png}
  \caption{}
  \label{fig:macro}
\end{subfigure}
\scalebox{1.5}{$\boldsymbol{\longrightarrow}$}
\begin{subfigure}{.33\textwidth}
  \centering
  \includegraphics[width=.6\linewidth]{./figures/CrystalGrain.png}
  \caption{}
  \label{fig:meso}
\end{subfigure}%
\scalebox{1.5}{$\boldsymbol{\longrightarrow}$}
\begin{subfigure}{.33\textwidth}
  \centering
  \includegraphics[width=.6\linewidth]{./figures/micro.jpg}
  \caption{}
  \label{fig:micro}
\end{subfigure}
\caption{a) Macro, b) meso, c) micro \cite{Ozawa:ko5009}}
\label{micmacmes}
\end{figure}


%generally determined by experiments made on bulk materials. These are materials at the macroscopic level. However in the case of heterogeneous %materials, these properties are the result of statistical averages of physical properties at a smaller level.
%These are of

Bleh bleh



If one correctly models the physical laws that are relevant at the mesoscale the properties at the macroscale should be extractible as the model gets large enough for the statistical differences in the  of  to average out.  Such a model is known as a \textit{representative volume element} or in short RVE.

Being able to model the materials at the mesoscale is useful since it enables us to see how changes in mesoscopic properties affects the macroscopic material. \textbf{Något om att det möjliggör material optimization?}



In this thesis the Voronoi tessellation is used to generate a microstructure which exhibit similar distribution in grain size as many polycrystalline materials. The generation of the tessellation and the meshing of it is done using thee open source program Neper.

The resulting mesh was then used to conduct a an analysis trying to simulate a uniaxial stress test on crystal plasticity

An introduction to cohesive elements in Abaqus is given... These elements are used to simulate the grain boundary laws between grains...
Cohesive elements were inserted between grain boundaries and test were done blah blah




\end{document}
