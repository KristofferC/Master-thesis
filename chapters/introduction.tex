% Sample file on how to use subfiles.
\documentclass[introduction.tex]{subfiles}

\begin{document}

\chapter{Introduction}

Multiscale -> macroscopig = sum mesoscopic


The purpose of this thesis is to grain structure with confomring mesh

Look at crystal plasticity

Then look at cohesive elements
- Lots of stuff here


Finally try it on WC/Co!

First done by 

\textbf{Stuff about WC/Co, where it is used, previous work}.

Many materials

In for example polycrystalline materials di


generally determined by experiments made on bulk materials. These are materials at the macroscopic level. However, these properties
are the result of 

If one correctly models the physical laws that are relevant at the mesoscale the macroscale properties should 

In this thesis the Voronoi tesselation is used to generate a microstructure which exihibit similar distribution in grain size as many polycrystalline materials. 
The generation of the tesselation and the meshing of it is done using thee open source program Neper.
 and mesh the resulting microstructure with a mesh conforming to the grain boundaries. A script was written to parse this mesh
 and insert new element between the grains. These elements are used to simulate the 

%Finite element analysis of this was done using the commercial software Abaqus of . 
%aim
input
output

parameter study



\end{document}
