% Sample file on how to use subfiles.
\documentclass[introduction.tex]{subfiles}

\begin{document}

\chapter{Introduction}



 \begin{figure}
\centering

\begin{subfigure}{.1\textwidth}
  \centering
  \includegraphics[width=.2\linewidth]{./figures/macro.png}
  \caption{}
  \label{fig:macro}
\end{subfigure}
\scalebox{1.5}{$\boldsymbol{\longleftarrow}$}
\begin{subfigure}{.33\textwidth}
  \centering
  \includegraphics[width=.6\linewidth]{./figures/CrystalGrain.png}
  \caption{}
  \label{fig:meso}
\end{subfigure}%
\scalebox{1.5}{$\boldsymbol{\longleftarrow}$}
\begin{subfigure}{.33\textwidth}
  \centering
  \includegraphics[width=.6\linewidth]{./figures/micro.jpg}
  \caption{}
  \label{fig:micro}
\end{subfigure}
\caption{Material properties at a certain scale are statistical averages of heterogeneous physical properties on a lower scale. The figure show a polycrystalline material at its a) macro scale, b) meso scale\cite{wiki:grain} and c) micro scale\cite{Ozawa:ko5009}}
\label{micmacmes}
\end{figure}


Most materials that on a macroscopic scale can be considered to be homogeneous are usually heterogeneous on a lower scale. The heterogeneous properties of such a material tend to average out giving a homogeneous appearance on the macroscopic level. In applications, materials with certain macroscopic properties might be desirable and knowledge how to create a material with these specific properties is valuable. 

By modelling the physical laws that are relevant at the mesoscale the correct macroscopic properties of the material should be possible to extract as the model gets large enough for the statistical differences in the heterogeneities to average out. Such a model is known as a \textit{representative volume element} or in short \textit{RVE}. Having such a model enables us to see how changes in mesoscopic properties affect the macroscopic material. Simulations can then be performed to predict which mesoscopic properties that gives the best macroscopic material for the given application. This is known as \textit{material optimization}. This reasoning gives the motivation for investigating methods in accurately modelling materials at different scales and how the scales are coupled. This is known as \textit{multiscale modelling} cf. \cite{zohdi}.

In this thesis the Voronoi tessellation algorithm (see for example \cite{voro}) is used to generate three dimensional microstructures which approximate the distribution of grain size and grain shape that is found in many polycrystalline materials. To determine the mechanical behaviour of the microstructure model, i.e. the RVE, finite element analyses should be performed. These will give us information about the response of the RVE but also how stresses and strains distribute within the RVE. The Finite Element (FE) analyses require that the RVE is discretized in space i.e. meshed. The meshing is performed with the open source program Neper \cite{Quey20111729}.  To investigate the performance of this methodology FE analyses of a two-phase stainless steel where the bulk behaviour of the two phases were modelled by crystal plasticity were performed. In these analyses the RVE was subjected to uniaxial tensile loading.

In order to have the capability of including grain boundary mechanisms in the model of the microstructure one method of doing so is examined. This method is the use of cohesive elements available in the commercial finite element analysis software Abaqus. Cohesive elements are inserted into the model of the microstructure at grain boundaries and similar finite element analyses as for the two phase steel are now performed for the prototype material tungsten carbide where cobalt has dissipated into the grain boundaries. The data for the traction separation law was taken from atomistic simulations performed in \cite{Gren2013}. It should then be possible to investigate how the boundary mechanisms affect the macroscopic response of a WC-Co alloy.

\end{document}
