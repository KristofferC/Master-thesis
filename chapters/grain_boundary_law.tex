% Sample file on how to use subfiles.
\documentclass[grain_boundary_law.tex]{subfiles}

\begin{document}

\FloatBarrier

\chapter{Grain boundary mechanisms}


In polycrystalline materials different mechanisms in the boundary between two grains can have major effects on the macroscopic behavior of the material. Investigations of grain boundary effects in different materials can for example be found in \cite{eff_grain}, \cite{grain_poly}. In order to improve the modeling of materials on a mesoscopic scale there is a need to include these effects into for example a finite element model. This is done in this thesis by something called \textit{cohesive elements}.

\section{Cohesive elements}

The finite element analysis software Abaqus, used in this thesis, has support for cohesive elements. In the manual for Abaqus it says that these "are primarily intended for bonded interfaces where the interface thickness is negligibly small.". In this thesis they are used to model the grain boundary deformation mechanisms that can exist between grains.

The difference between cohesive elements and solid elements are that cohesive elements do not model any material but instead they model the actual interface between two material surfaces.Since the cohesive elements do not contain any material there is no stress tensor like in solid elements. Instead when the cohesive element is subjected to deformations then  but instead a vector of the traction components, the normal traction $t_n$ and two shear tractions $t_s$ and $t_t$ is the output. The normal direction is defined from the order of the nodes defining the vertices of the element as shown in figure \ref{fig:cohs_ori}. The corresponding separations are denoted by $\delta_n, \delta_s, \delta_t$. The nominal strain in the element are defined as
%
\[ \epsilon_n = \frac{\delta_n}{T_0}, \quad \epsilon_s = \frac{\delta_s}{T_0}, \quad \epsilon_t = \frac{\delta_t}{T_0}  \]
%
The parameter $T_0$ is not the thickness of the cohesive element but a constitutive thickness that when set to 1.0 it yields the strain and the separation become equal. For a more detailed description of how the constitutive thickness works see section 31.5.6 in the Abaqus manual.



\begin{figure}[htpb!]
\centering
 \includegraphics[width=.4\linewidth]{./figures/cohs_ele}
\caption{Orientation of a cohesive element. The order of the nodes is used to determine the bottom and top face as well as the positive normal direction. In order for the normal direction to point upwards the nodes should be given in a counter clockwise direction.}
\label{fig:cohs_ori}
\end{figure}


When modeling damage and failure of a material,or rather an interface surface, by using cohesive element there are three main areas of interest:

\begin{enumerate}
\item The properties of the material interface before damage has taken place. Abaqus only supports a linear elastic traction-separation law prior to damage. This is however not a major limitation since many materials have a linear response before damage occurs. 

\item The criterion that defines when the material starts experiencing damage. This is called the \textit{damage initiation criterion}.

\item The properties of the material interface after that the damage initiation criteria has been fulfilled need to be prescribed. This is done by \textit{damage evolution} which describes how the degradation of the material stiffness develops as the material interface experience more and more separation.
\end{enumerate}
%
A possible scenario for the damage initiation and evolution can be seen in figure \ref{fig:tract_sep}.

\begin{figure}[htpb!]
\centering
  \includegraphics[width=.6\linewidth]{./figures/tract_sep}
\caption{An example of a traction separation law. The point $a$ is the damage imitation point where the stiffness of the material interface start to degrade. The damage evolution is the interval covered by $b$ showing how the damage is evolving with increasing separation. In this figure a linear damage evolution is assumed. At the separation point $\delta_n^t$ the material interface experience a complete failure and fracture occurs.}
\label{fig:tract_sep}
\end{figure}

\subsection{Undamaged behavior}
 
The behavior of the material before damage initiation is assumed to be elastic and is written in term of a constitutive matrix as
 %
 \[ \vec{t} = 
 \left[
\begin{array}{c c c}
t_n\\
t_s\\
t_t \\
\end{array} \right]
=
\left[
\begin{array}{c c c}
K_{nn} & K_{nt} & K_{ns} \\
K_{tn} & K_{tt} & K_{ts} \\
K_{sn} & K_{st} & K_{ss} \\
\end{array} \right]
 \left[
\begin{array}{c c c}
\epsilon_n\\
\epsilon_s\\
\epsilon_t \\
\end{array} \right]
= \vec{K} \vec{\epsilon}.
\]
%
As can be seen from the matrix it is possible to allow for a coupling between normal and shear components. However, in this thesis only uncoupled behavior will be used and thus only the diagonal elements of $K$ will be non zero in this analysis. Any difference between the shear directions will not be considered which means that $K_{tt} = K_{ss}$.

\subsection{Damage initiation}

The damage initiation criterion is the point when the degradation of the material interface starts to occur. This criterion can either be described as a function of the strain or the traction. There are two possible functions that can be used in Abaqus, one that is based on the maximum strain or traction component and one that is based on length of the strain or traction vector. For the traction criteria these functions can be expressed as
%
\[ \max \left( \frac{\langle t_n \rangle}{t_n^0} ,\frac{ t_s }{t_s^0} ,  \frac{ t_t }{t_t^0}  \right) = 1,  \] 

\[  \left( \frac{\langle t_n \rangle}{t_n^0} \right)^2 +   \left( \frac{ t_s }{t_s^0} \right)^2 +   \left( \frac{ t_t }{t_t^0} \right)^2 = 1.   \]	
%
Here $t_n^0, t_s^0, t_t^0$ are three user defined parameters. The symbol $\langle \rangle$ denotes the Macaulay brackets which is non zero only if the quantity inside the bracket is positive. This has the effect that pure compression (negative normal traction) do not lead to any damage. The corresponding functions for the strains are found by trivial modifications to the above functions. In the analysis used in this thesis the quadratic strain criterion is used. 

\subsection{Damage evolution}

The damage evolution describes the degradation of the material interface as a function of the separation. The current degradation of the material is completely captured by one scalar damage variable $D$. The traction vector is then calculated as 
%
\[ \vec{t} = (1 - D) \overline{\vec{t}} \]
%
 where $\overline{\vec{t}}$ is the traction in the element if there was no damage in the material interface. Before the damage initiation criterion has been fulfilled, $D = 0$. Having the damage completely described by only a scalar is one of the major limitations of the cohesive elements that are available in Abaqus. This means it is hard to capture the difference in the traction separation law for normal and shear separations. It would be useful if the damage in the material interface could be described by a three component vector or even more generally by a tensor. This is likely possible by writing a user defined element where one codes how the traction is calculated from the separations (known in Abaqus as a \texttt{UEL}) but doing this out of scope for this thesis.

If we first assume that we only have separations in the normal direction and want to give $D$ in table form as an input to Abaqus, the following procedure would be used. $K_{nn}$ is calculated as the slope of the initial elastic response. For separations larger than the damage initiation $\delta_n^0$,  at a specific point $(\delta_n, \sigma_n)$, $D$ is calculated as $D_f(K_{nn}, \delta_n, \sigma_n)$ where $D_f$ is the function:
%
\[ D_f(K, \delta, \sigma) \mapsto   1 - \frac{\sigma}{K\delta}. \]
%
However, in general, we do not have pure normal or pure shear separations. From Abaqus we can get a measure of the relation in magnitude between the normal and shearing traction components. This measure is a number $\phi_1$ between 0 and 1 that is defined as
%
\[ \phi_1 = \frac{2}{\pi} \arctan \left( \frac{ \tau}{\langle t_n \rangle} \right) \]
%
where $\tau$ is the effective shear traction $\sqrt{t_s^2 + t_t^2}$. It is also possible to get a number $\phi_2$ that gives a relation between the shear tractions but since we will not assume any difference between the shear directions in this thesis, this number is not used. Using the equation for $\phi_1$ together with the equation for the effective separation,
%
\[ d_m = \sqrt{d_n^2 + d_s^2 + d_t^2},  \]
%
we can solve for the normal separation $d_n$ and the effective shear separation $ d_\tau = \sqrt{d_s^2 + d_t^2}.$ in terms of the known $\phi_1$ and $d$. The following results are obtained:
%
\begin{align}
 d_n & = \frac{d_m}{\sqrt{\left(\frac{PK_{nn}}{K_{tt}}\right)^2 + 1 }  }  \\
 d_\tau  & = d_m\sqrt{1 -\frac{1}{\left(\frac{PK_{nn}}{K_{tt}} \right)^2 + 1 }}.
\end{align}
%
We now calculate two different damage scalars, $D_n = D_f(K_{nn}, \delta_n, \sigma_n)$ and $D_\tau = D_f(K_{tt}, d_\tau, \sigma_\tau)$. These are the correct damage scalars for the case of pure shear or pure normal separation by a distance of $\delta_n$ or $\delta_\tau$ respectively. The issue now is how these two damage scalars should be combined to give an overall damage of the interface. One way of doing this is by adopting the formula,
%
\[ D = 1 - (1 - D_n)(1 - D_t). \]
%
For mixed separation conditions, $D$ is always smaller than the individual $D_n$ and $D_t$ This formula is similar to how Abaqus calculates the total damage when multiple damage criteria are active, see section 23.2.3 in the Abaqus manual. 


\subsection{Insertion of cohesive elements into mesh}



From Neper we get a mesh of the generated microstructure but Neper does not have the capability of inserting cohesive elements between the grains. Hence, this must be done as a post processing of the generated mesh. A script was written in the programming language Python that parsed the mesh file and inserted the cohesive elements. In order to facilitate the description of the algorithm the script uses to insert the cohesive elements, some notation will be given here. 

\begin{itemize}
\item A node $n$ is represented by a unique identifier $i \in \mathbb{N}$ among the nodes and three coordinates indicating its location in space $x_i,y_i,z_i$. It is denoted as $n_i(x_i,y_i,z_i)$.
\item A first order triangle element $T$ is represented by a unique identifier $j \in \mathbb{N}$ among all the elements and the three nodes that are used as its vertices. It is denoted as $T_j(n_{j_1}, n_{j_2}, n_{j_3})$. 
\item A cohesive element $C$ also has a unique identifier but instead of being defined by nodes it is defined in terms of the two triangles which prescribes the two faces perpendicular to the normal of the cohesive element, it is written as $C_j(T_{j_1}, T_{j_2})$.
\end{itemize}
The algorithm can now be described in two main steps. The algorithm is described for one grain boundary but the same method is used for all boundaries.
\begin{enumerate}
\item \textit{Disjoin the shared face in a grain boundary.} - In the generated mesh each face is shared by two grains except the faces on the boundary of the cube which are ignored. An example of two grains with a shared face can be seen in \ref{fig:cohs_2_a}. What we want to do is create two separate faces, one for each grain, like in figure \ref{fig:cohs_2_a}. From the Neper mesh file, element sets with triangles in the current face can be extracted. The following method is used to separate the faces between two grains. To identify the grains they are assigned a number $G = 1,\ldots , n_g$ where $n_g$ is the total number of grains in the microstructure. The identifiers for the two grains on the current grain boundary will be denoted $G_1$ and $G_2$. From each node $n_i (x_i, y_i, z_i)$ in the face two new nodes are created, $n_{f(G_1,i)} (x_i, y_i, z_i)$ and $n_{f(G_2,i)} (x_i, y_i, z_i)$. These new nodes need unique identifiers and this is the purpose of the function $f$. The function $f$ should take an identifier for a grain and a node such that:
%
\begin{align}&  f: (\mathbb{N},\mathbb{N}) \rightarrow \mathbb{N} \\
 & f(G_1, i) = f(G_2, j) \Leftrightarrow (G_1, i) = (G_2, j) 
\end{align}
%
If this property does not hold then the newly created nodes might have the same identifier as other nodes which would cause problems since each identifier needs to be unique. One simpler way of always obtaining a unique identifier for the new nodes it to increment the identifier by one each time a new node is added. However, this does not work since nodes in the edges of grains are part of multiple faces. Duplicated nodes for these faces would have different identifiers even though they are part of the same grain and originating from the same node. This would cause hanging nodes. A suitable function which was used in this thesis is
%
\begin{align} 
  f(G, i) = G \cdot n_\text{tot} + i,
\end{align}
%
where $n_\text{tot}$ is the total number of nodes in the original mesh. This has the effect that some of the identifiers for the newly created node can become quite large. However, the identifiers are sparse in the sense that there are many numbers between different identifiers that are not used by any other node as an identifier. This means that after the insertion of cohesive elements are done, then the identifiers can be renumbered in such a way that they are dense and at this point they are no longer a higher number than necessary.

With the new nodes in the mesh, new triangles can be defined. From the set of triangles in the face $\{ T_j(n_{j_1}, n_{j_2}, n_{j_3}) | \forall j: T_j \in \text{face} \}$, two new sets are created, one for each grain 
\[ \{ T_{j_1}(n_{f(G_1,j_1)}, n_{f(G_1,j_2)}, n_{f(G_1,j_3)}) | \forall j: {T_{j} \in \text{face} \}} \]
 and 
\[ \{ T_{j_2}(n_{f(G_2,j_1)}, n_{f(G_2,j_2)}, n_{f(G_2,j_3)}) | \forall j: {T_{j} \in \text{face} \}}.\]  
 These two sets of triangles represent two new faces that do not share any nodes but are at the same location in space as the original face. What needs to be done now is to connect the tetrahedral element that where part of the original face to the newly created faces. This is done by searching each tetrahedral inside the first and second grain, with identifier $G_1$ and $G_2$ respectively, for nodes $n_i \in \text{face}$ and swap these nodes for the newly created nodes  $n_{f(G_1, i)}$ and $n_{f(G_2, i)}$ respectively.

\item \textit{Connect the new faces with cohesive elements}. At this point we have two faces that have no connection of elements to each other. These are now easily connected by creating the set of cohesive elements $ \{ C_j(T_{j_1},T_{j_2}) |  \forall j: T_j \in \text{face}  \} $. This is illustrated in figure \ref{fig:cohs_2_c}. 
\end{enumerate}


\begin{figure}[htpb!]
\centering
\begin{subfigure}[b]{.5\linewidth}
  \centering
  \includegraphics[width=.5\linewidth]{./figures/2_cohs}
  \caption{}
  \label{fig:cohs_2_a}
\end{subfigure}%
\hspace{-10mm}
\begin{subfigure}[b]{.5\linewidth}
  \centering
  \includegraphics[width=.5\linewidth]{./figures/2_grain_no_cohs}
  \caption{}
  \label{fig:cohs_2_b}
\end{subfigure}%
\hspace{-10mm}
\begin{subfigure}[b]{.5\linewidth}
  \centering
  \includegraphics[width=.5\linewidth]{./figures/2_grain_cohs}
  \caption{}
  \label{fig:cohs_2_c}
\end{subfigure}
\caption{Insertion of cohesive elements between two grains. (a) The two grains. (b) Duplication of nodes in the face. Grains have been separated for clarity of the figure. (c) Cohesive elements inserted between the two faces. }
\label{fig:cohs_2}
\end{figure}

After this algorithm has been executed for all faces the cohesive elements are inserted. The new mesh can then be exported and used in FE analyses. A figure where cohesive elements have been inserted between grains in a full microstructure can be seen in figure \ref{fig:cohs_large}. In the figure the grains have been pulled apart from each other in order for the cohesive elements to have a finite width and be visible.



\begin{figure}[htpb!]
\centering
\begin{subfigure}[b]{.5\textwidth}
  \centering
  \includegraphics[width=.7\linewidth]{./figures/cohs_with_poly}
  \caption{}
  \label{fig:cohs_large_a}
\end{subfigure}%
\begin{subfigure}[b]{.5\textwidth}
  \centering
  \includegraphics[width=.7\linewidth]{./figures/cohs_without_poly}
  \caption{}
  \label{fig:cohs_large_b}
\end{subfigure}
\caption{Cohesive elements with a) grains shown and b) grains hidden.}
\label{fig:cohs_large}
\end{figure}


\section{FE analyses with cohesive elements}

In order to test the methos well as FE analyses were performed on a micro structure with material properties that were not set to simulate a real case.

One Voronoi tesselation of 50 grains was generated and meshed. Cohesive elements were then inserted into the mesh. The grains were set to behave elastically with a Young's modulus of $E = \unit[200]{GPa}$. The cohesive elements were set to initially have be mode independent with an elasticity equal to the one in the solid elements. The damage initiation criteria was $\sigma_n^0 = 0.001$ and a linear damage evolution law was used with fracture at $\sigma_n^t = 0.04$. The same method of inducing strain as in section \ref{chap:cryst_plast} was used. The analysis was run for meshes of different coarseness to see if there was any mesh dependence in the results. The resulting strain stress curves can be seen in figure \ref{fig:coarse_coh}.
%
%
\begin{figure}[htpb!]
\centering
\includegraphics[width=.7\linewidth]{./figures/mesh_course_coh}
\caption{Courseness mesh}
\label{fig:coarse_coh}
\end{figure}
%
 The analyses was set to run until a macroscopic strain of 0.05 of the RVE. However, at different points in the analysis convergence problems with the implicit solver caused the analysis to abort. This happened at different points during the analysis. Implicit solvers generally have difficulties with softening in materials and the discontinuous nature in the gradient of the traction separation law in cohesive elements probably make things worse. It is likely that there is a discontinuity in the force displacement curve causing the Newton Rhapson method to not converge to an equilibrium point.A common advice to overcome the convergence problems with implicit solvers associated with the softening of materials due to cohesive elements is to introduce viscous regularization in the cohesive elements. The viscous stiffness degradation variable $D_v$ is defined as
 
\begin{equation}
  \dot{D}_v = \frac{1}{\mu} (D - D_v)
\end{equation}

where $\mu$ is the viscosity parameter. The response of the damaged cohesive interface is then given as
%
\begin{equation}
  \vec{t} = (1 - D_v) \overline{\vec{t}}.
\end{equation}
%
This should have the effect of smoothing out the damage in the cohesive element making it easier for the implicit solver to find an equilibrium point. This was attempted but it had little to no effect in terms of convergence. The exact criterion that causes the convergence problem is yet unknown.

There are however another way of solving the FE problem. The implicit solver tries to solve the equilibrium equation
%
\begin{equation}
  \vec{K} \vec{u} - \vec{R} = 0,
\end{equation}
%
for notation see section \ref{sec:femmech}. Instead of solving the equilibrium equation we can take into account acceleration giving the equation
%
\begin{equation}
 \vec{M}\vec{\ddot{u}} + \vec{K} \vec{u} - \vec{R} = 0,
\end{equation}
%
where $\vec{M}$ is known as the  \textit{mass matrix} defined as
%
\begin{equation}
M_{ij} = \int_\Omega \varphi v_i v_j \id x
\end{equation}
%
This can be solved by for example the central different method as
%
\begin{equation}
\frac{1}{\Delta t^2} \vec{M} \vec{u}^{t + \Delta t} = \vec{R}^t - \vec{K} \vec{u}^t + \frac{1}{\Delta t^2} \left(2 \vec{M} \vec{u}^{t} - \vec{u}^{t-\Delta t} \right)
\end{equation}
This is however conditionally stable on the time step $\Delta t$ used, c.f \cite{Park1977343}. A necessary condition for convergence is the  Courant–Friedrichs–Lewy condition. In FE analyses this condition can be stated that he distance traveled by the fastest wave in the material in one time step should be smaller than the characteristic size of the element $L$. The speed of the waves are related to the stiffness $k$ and mass $m$ of the material. The condition can be written as
\begin{equation}
\Delta t \leq \alpha \frac{L \sqrt{m}}{\sqrt{k}}
\end{equation}

This condition can be quite strict and a small timestep might thus be needed leading to extensive computational time required. However, the problem we are trying to solve is static and should be independent of the mass. We should therefore be able to scale the mass in the FE model in order for the larger time steps to respect the convergence condition without significantly affecting the result. In order to study this analyses were the mass was scaled in order for the stable time step to change were perfomed and the results are shown in figure \ref{fig:mass_scale}. It can be seen if the mass is scaled high enough the resulting stress is irregular and seem to have some high frequency oscillations. However, with smaller time steps the curves seem to converge nicely. 

\begin{figure}[htpb!]
\centering
\includegraphics[width=.7\linewidth]{./figures/exptime}
\caption{Mass scaling for a stable time increment of $1e-03,1e-04,1e-05$ .}
\label{fig:mass_scale}
\end{figure}

In order to test the validity of the result from the explicit solver the results are compared to the implicit solver for identical mesh and material properties. In figure \ref{fig:impvsexp} the results are shown and very good agreement between the explicit and implicit solutions can be seen. From this we draw the conclusion that it is possible to use the explicit solver in Abaqus in order to circumvent the convergence problems associated with the implicit solver.

\begin{figure}[htpb!]
\centering
\includegraphics[width=.7\linewidth]{./figures/impvsexp}
\caption{Different}
\label{fig:impvsexp}
\end{figure}



\end{document}