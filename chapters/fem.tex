% Sample file on how to use subfiles.
\documentclass[fem.tex]{subfiles}

\begin{document}

\chapter{Introduction to finite element method}

Here follows a short introduction to the basic principles behind the Galerkin finite element method for elliptic problems. This is later put into the context of a 

\begin{enumerate}
\item Reformulate the problem as a weak formulation.
\item Discretize the problem. This is done by constructing a finite dimensional space 
\item Solve the discrete problem.

\end{enumerate}

\section{Galerkin method}
\label{sec:galerkin}
Assuming that we have the problem in it's weak form , that is 

\[  a(u,v) = f(v) \quad \forall v \in X \]

The space $\Omega$ is then discretized into a finite dimensional subspace $\Omega_h$ of dimension $d$ with the basis $\{e_1, \ldots, e_d\}$. The function $v$ can now be written as a linear combination of the basis

\[ v = \sum_{i=1}^d n_i e_i, \quad n_i \in \mathbb{R} \]

The discrete representation of the problem, where we have moved from the space $X$ to the discrete space $X_h$, is formulated as
%
\begin{equation} 
    a(u_h, v) = f(v) \quad \forall v \in X_h
    \label{eqn:disc_prob}
\end{equation}
%
and in particular
\begin{equation} 
    a(u_h, e_j) = f(e_j) \quad j = 1, \ldots, d
    \label{eqn:disc_part}
\end{equation}

Representing $u_h$ in the new basis gives
\begin{equation} 
    a \left(\sum_{i=1}^d \eta_i e_i, e_j\right) = f(e_j) \quad j = 1, \ldots, d, \quad \eta_i \in \mathbb{R}
    \label{eqn:u_new_base}
\end{equation}

Exploiting the bilinearity in $a$ this can be rewritten as 
\begin{equation} 
    \sum_{i=1}^d a \left( e_i, e_j\right)\eta_i = f(e_j) \quad j = 1, \ldots, d, \quad 
    \label{eqn:final}
\end{equation}

This can in condensed form be written as
\begin{equation} 
    A \eta = f
    \label{eqn:cond}
\end{equation}
where $A_{ij} = a(e_i, e_j)$ and $f_j = f(e_j)$. $A$ is commonly known as the \textit{stiffness matrix} and 

This system of equations can then be solved by direct or iterative methods.%


\section{fhdsif}
In the structural mechanical problems the momentum equation is solved
%
\begin{equation} 
   \sigma_{ji,j} + f_i = 0
    \label{eqn:cond}
\end{equation}
%

%
\begin{equation} 
  \int_\Omega  v_i\sigma_{ji,j} + \int_\Omega v_i f_i = 0
    \label{eqn:cond}
\end{equation}
%

%
\begin{equation} 
  \int_\Omega  v_{i,j} \sigma_{ji}  = \int_\Gamma v_i t_i + \int_\Omega v_i f_i
    \label{eqn:weak}
\end{equation}
%
where traction vector $t_i = \sigma_{ij}n_j$ has been introduced. This is the weak formulation of the original problem. It can be shown that this has the same solutions.

By using a constitutive relations and a discritization of the space we can follow the same method as described in the section \ref{sec:galerkin}, an equation of the same type as  \ref{eqn:final} can be written. As an example if we assume linear elasticity we can write

\[ \sigma_{ij} = C_{ijkl} \epsilon_{kl} \quad  \epsilon_{kl}(u) = \frac{1}{2} \left(u_{k,l} + u_{l,k}\right)  \]

\[ v_{i,j} \sigma_{i,j} =  \epsilon_{ij}(v) \sigma_{i,j} \]

By expanding $v$ into $\sum n_i e_i$ and $b_i = n_{i,j}$  = 

\begin{equation} 
  \int_\Omega  \epsilon_{ij}(v)  C_{ijkl} \epsilon_{kl}(u)  = \int_\Gamma v_m t_m + \int_\Omega v_m f_m
    \label{eqn:weak}
\end{equation}
%

By expanding $v$ and $u$ into a discritized base and running through the same argument as in section we end up with the expression:

\begin{equation} 
\sum_{o = 1}^d \left( \int_\Omega  \epsilon_{ij}(N_p)  C_{ijkl} \epsilon_{kl}(N_o)\right) a_{om}  = \int_\Gamma N_p t_m + \int_\Omega N_p f_m
    \quad p = 1, ..., D, \quad m = 1,2,3
\end{equation}

\begin{equation} 
A_{po} a_{om} = f_{pm}
\end{equation}


\end{document}