% Sample file on how to use subfiles.
\documentclass[wcco.tex]{subfiles}

\begin{document}

\FloatBarrier


\section{FE analysis of Tungsten Carbide with Cobolt dissipated }


\textbf{This is WC-CO and should be moved to a separate chapter}

The way Cobolt (Co) and Tungsten Carbide (WC) is assigned to different grains is as follows. The smallest grains are set to contain Cobolt until they constitute the desired volume fraction. The rest of the grains are set to Tungsten Carbide. Cohesive elements are then inserted in the faces that are shared by either two WC grains or one WC grain and one Co grain. There are thus no cohesive element between two Co grains. The reasoning behind this is that the Co grains are usually smaller than the WC grans.  An example of the assignments of materials to a microstructure can be seen in figure \ref{fig:cowc}.



In order to confirm that the data was entered correctly into Abaqus a simple test was done. The test was one with a model of two grains with the shape of rectangular prism. The grains were connected with cohesive elements and the top grain was displaced in different directions. The directions used where  In this test the grains were made very stiff such that only deformation in the cohesive elements took place. The model after displacement can be seen in figure \ref{2_grain_test_case}. The two white grains can be seen as well as the green cohesive elements connecting them together. The displacement has been exaggerated for clarity of the figure.

\begin{figure}[ht]
\centering
\includegraphics[width=.3\linewidth]{figures/2_grain_test_case}
\caption{The test case}
\label{fig:test_case}
\end{figure}


\begin{figure}[ht]
\centering
\includegraphics[width=.75\linewidth]{figures/abaq_coh_test}
\caption{Results for the traction in the cohesive zone elements for three different directions. It can be seen that the traction law with pure normal and pure shear displacement is respected and reasonable results are shown for mixed separations. }
\label{fig:test_cases}
\end{figure}


\end{document}