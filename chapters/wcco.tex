% Sample file on how to use subfiles.
\documentclass[wcco.tex]{subfiles}

\begin{document}

\FloatBarrier


\section{FE analyses of tungsten carbide}

To test the cohesive element method on a real world material FE analyses were performed on a model where the characteristics of the traction separation law were based on results from atomistic simulations in \cite{Gren2013}. The traction separation law for the grain boundaries in tungsten carbide was entered in the cohesive elements and FE analyses were performed.
 
\subsection{Test case} 
  In order to confirm that the data was entered correctly into Abaqus a simple test was performed. The test was defined by a model of two grains with the shape of rectangular prism. The grains were connected with cohesive elements and the nodes in the top grain was displaced in different directions. In this test the grains were made very stiff such that only deformation in the cohesive elements took place. The model after displacement can be seen in figure \ref{fig:2_grain_test_case}. The two white grains can be seen as well as the green cohesive elements connecting them. The displacement has been exaggerated for clarity of the figure.

\begin{figure}[ht]
\centering
\includegraphics[width=.3\linewidth]{figures/2_grain_test_case}
\caption{The model that was used for the test case.}
\label{fig:2_grain_test_case}
\end{figure}

The atomistic simulations were only performed with displacements of the bulk grains in pure normal and shear directions. There is therefore a need to interpolate the traction separation law between these two cases and to do this the method described in section \ref{sec:dmg_evo} was used. The top grain in the test case was displaced along the pure normal and shear directions as well as in a mixed direction. The traction in the cohesive elements for the different directions is shown in figure \ref{fig:test_cases}. The traction separation law for the edge cases are exactly those that was entered into the cohesive element and the result for the mixed case is sensible. This confirms that the method used to interpolate the traction separation law is reasonable and that the pure normal and shear responses are corret.


\begin{figure}[ht]
\centering
\includegraphics[width=.75\linewidth]{figures/abaq_coh_test}
\caption{Results for the traction in the cohesive zone elements for three different directions. The traction law with pure normal and pure shear displacement is respected and reasonable results are shown for mixed separations. }
\label{fig:test_cases}
\end{figure}


\subsection{Full analysis}

The same traction separation law as in the test case above was used for these analyses of three dimensional grain structures. The grains of tungsten carbide were modelled to deform elastically with E $\approx \unit[250]{GPa}$. This is consistent with results in \cite{Buss04}.
This effects of the cobalt skeleton that usually exist in this material is not taken into account in this thesis. The size of the grains were set to average 1 micron. Simulations were done with 50, 100, 250, 500 grains. Results of the average stresses can be seen in figure \ref{fig:wcco_res}

\begin{figure}[ht]
\centering
\includegraphics[width=.7\linewidth]{figures/avg_fig_wcco.pdf}
\caption{Results for wolfram carbide for different number of grains}
\label{fig:wcco_res}
\end{figure}


\section{Discussion}
Some more stuff here.
\begin{itemize}
\item A lot of things happen at a small time scale when the grain fractures. In later analyses more care must be taken to have a better resolution at that part of the run.
\item These are recent results, more simulations must be done. For example it can be seen that the fracture is not as abrupt at 500 grains than 50. It is possible that higher number of grains is needed.
\end{itemize}



\end{document}