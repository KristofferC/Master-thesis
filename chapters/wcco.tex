% Sample file on how to use subfiles.
\documentclass[wcco.tex]{subfiles}

\begin{document}

\FloatBarrier


\section{FE analysis of Tungsten Carbide with Cobolt between grains }

In order to test the cohesive element method on a real world material FE analyses were performed on a model with a traction separation law

\subsection{Tractin separation law of WC-Co}

Atomistic simulation were performed in \textbf{[martin]} and one of the traction separation law were entered in the cohesive elements. In order to confirm that the data was entered correctly into Abaqus a simple test was done. The test was one with a model of two grains with the shape of rectangular prism. The grains were connected with cohesive elements and the top grain was displaced in different directions. The directions used where  In this test the grains were made very stiff such that only deformation in the cohesive elements took place. The model after displacement can be seen in figure \ref{2_grain_test_case}. The two white grains can be seen as well as the green cohesive elements connecting them together. The displacement has been exaggerated for clarity of the figure.

\begin{figure}[ht]
\centering
\includegraphics[width=.3\linewidth]{figures/2_grain_test_case}
\caption{The test case}
\label{fig:2_grain_test_case}
\end{figure}


\begin{figure}[ht]
\centering
\includegraphics[width=.75\linewidth]{figures/abaq_coh_test}
\caption{Results for the traction in the cohesive zone elements for three different directions. It can be seen that the traction law with pure normal and pure shear displacement is respected and reasonable results are shown for mixed separations. }
\label{fig:test_cases}
\end{figure}

\subsection{Bulk behaviour of WC}

Elastic with $E = 250GPa$ from Buss thesis.


\section{Load case}
Size of grains on average 1 micron. Simulations were done with 50, 100, 250, 500 grains. Results of the average stresses can be seen in figure \ref{fig:wcco_res}

\begin{figure}[ht]
\centering
\includegraphics[width=.7\linewidth]{figures/avg_wcco.png}
\caption{Results for wolfram carbide for different number of grains}
\label{fig:wcco_res}
\end{figure}


\section{Discussion}
\begin{itemize}
\item A lot of things happen at a small time scale when the grain fractures. In later analyses more care must be taken to have a better resolution at that part of the run.
\item These are recent results, more simulations must be done. For example it can be seen that the fracture is not as abrupt at 500 grains than 50. It is possible that higher number of grains is needed.
\end{itemize}



\end{document}