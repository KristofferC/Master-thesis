% Sample file on how to use subfiles.
\documentclass[generate_interface_elements.tex]{subfiles}

\begin{document}

\chapter{Generation of interface elements}

From the mesh file generated by Neper a script was written that inserted interface elements in the faces between grains. This is done by:


Maybe some pseudocode here?
\begin{itemize}
\item  Loop over every triangle in the face.
\item  Make two copies of the node in the triangle.
\item  Make a new interface element from the six new nodes.
\item  Attach all elements in grain1 that previously was connected to the triangle in the face to instead connect to the nodes of one of the sides of the newly created interface element.
\item  Do the same for the elements in grain 2.
\end{itemize}




An illustration in 2D displaying the interface elements after two grains are pulled apart can be seen in in figure \ref{fig:interface_elements}.

\begin{figure}
\centering
\includegraphics[scale=0.5]{figures/interface_elements}
\caption{Interface elements shown in red. Not very good figure again. Maybe redo with less ugly colors and three grains. Alternatively take image from Abaqus before and after negative pressure}
\label{fig:interface_elements}
\end{figure}

\newpage

\end{document}

