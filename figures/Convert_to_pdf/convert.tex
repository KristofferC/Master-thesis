\documentclass[border=12pt, crop]{standalone}
    \usepackage{psfrag}
    \usepackage{graphicx}
    \pagestyle{empty}
  
    % You can run this by typing the following commands:
    %    latex 2dplot.tex
    %    dvips -o 2dplot.ps 2dplot.dvi
  
    \begin{document}
  
    % The syntax of the "psfrag" command is:
    %    \psfrag{tag}[<posn>][<psposn>][<scale>][<rot>]{replacement}
    % See the file pfgguide.ps for full documentation.
  
      \psfrag{ngrain}[l][B][1][0]{Average total force among the runs}
      \psfrag{ngrain5}[B][B][1][0]{$n = 5$}
      \psfrag{ngrain20}[B][B][1][0]{$n = 20$}
      \psfrag{ngrain50}[B][B][1][0]{$n = 50$}
      \psfrag{ngrain100}[B][B][1][0]{$n = 100$}
      \psfrag{ngrain250}[B][B][1][0]{$n = 250$}
      \psfrag{ngrain500}[B][B][1][0]{$n = 500$}
      \psfrag{Total force}[B][B][1][0]{Total force}
      \psfrag{Displacement}[B][B][1][0]{Displacement $u_y$}
            
      \psfrag{uy}[l][l][1.5][0]{$u_y(t)$}
      \psfrag{uy=0}[l][l][1.5][0]{$u_y = 0$}
      \psfrag{uz=0}[l][l][1.5][0]{$u_z = 0$}
      \psfrag{ux=0}[l][l][1.5][0]{$u_x = 0$}
 \psfrag{x}[B][B][1][0]{$x$}
  \psfrag{y}[B][B][1][0]{$y$}  
  \psfrag{z}[B][B][1][0]{$z$}
  
    \psfrag{tn}[B][B][1][0]{$t_n$}
        \psfrag{dn0}[B][B][1][0]{$\delta_n^0$}

    \psfrag{dnt}[B][B][1][0]{$\delta_n^t$}
    \psfrag{a}[B][B][1][0]{$a$}
    \psfrag{b}[B][B][1][0]{$b$}
     \psfrag{stress}[B][B][1][180]{$\overline{\sigma}_{yy}$ [MPa]}
   \psfrag{eps}[B][B][1][0]{$\overline{\epsilon}_{yy}$}

   \psfrag{5grains}[B][B][1][0]{$5 grains$}
 \psfrag{20grains}[B][B][1][0]{20 grains}
  \psfrag{50grains}[B][B][1][0]{50 grains}
   \psfrag{100grains}[B][B][1][0]{100 grains}
    \psfrag{200grains}[B][B][1][0]{200 grains}
    \psfrag{500grains}[B][B][1][0]{500 grains}
      \includegraphics[scale = 0.5]{mesh_coars_imp.eps}

    \end{document}